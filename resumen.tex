\textit{Staphylococcus aureus} es un patógeno de importancia clínica que ha recibido gran atención pública debido al desarrollo de resistencia a diferentes antibióticos tradicionales dirigidos a inhibir la síntesis de la pared de péptidoglicando que recubre la bacteria. Como alternativa de tratamiento se está estudiando el mecanismo de acción de los péptidos antimicrobianos dirigidos a comprometer la integridad de la membrana lipídica de la bacteria. Los péptidos antimicrobianos atacan a la membrana plasmática de la bacteria formando poros, los cuales disipan el potencial electroquímico requerido para mantener viva la bacteria. Es importante estudiar las propiedades mecánicas de la membrana plasmática ya que la bacteria puede modular estas propiedades a través de cambios en su composición lipídica, lo cual puede resultar en incrementos en resistencia a péptidos antimicrobiales. Uno de los compuestos que se ha estudiado en \textit{Staphylococcus aureus} es  el carotenoide estafiloxantina, el cual aumenta  su concentración en la membrana plasmática cuando la bacteria está sometida a estrés. Algunos estudios han sugerido que el papel de este carotenoide es aumentar la rigidez de la membrana plasmática de la bacteria. Sin embargo, aún no está clara la función que cumple en la membrana asociada a sus propiedades biofísicas locales, propiedades tales como la orientación y las interacciones con lípidos cercanos. Uno de los métodos que puede contribuir al estudio del papel local de la estafiloxantina en la membrana es el de las simulaciones por dinámica molecular. En estas simulaciones se simula la trayectoria de los átomos que conforman cada molécula de un fragmento pequeño (128 lípidos y sus moléculas de agua asociadas) de la membrana plasmática.
Debido al vacío existente en el conocimiento de estafiloxantina en \textit{Staphylococcus aureus} es objeto del presente trabajo estudiar el rol mecánico de la estafiloxantina al insertarse en dos membranas modelo de \textit{Staphylococcus aureus}: una que contiene DMPG y otra que contiene DPPG. El rol mecánico de la membrana se estudiará mediante sumulaciones por dinámica molecular en las cuales se usarán los campos de fuerza de AMBER y de CHARMM. En estos campos se optimizarán los parámetros de los ángulos dihedros de los diferentes grupos moleculares de estafiloxantina para tratar de reflejar el comportamiento físico de la molécula. De las simulaciones se pueden obtener propiedades biofísicas como el parámetro de orden del deuterio, la orientación de la estafiloxantina, el área por lípido, entre otras. Estas propiedades pueden ser utilizadas para predecir cambios en las propiedades mecánicas de la membrana en presencia de esta molécula.
