\section{Objetivo Principal}
Obtener campos de fuerza para estafiloxantina optimizando los parámetros de los ángulos dihedros de la cadena diaponeurosporenoica y estudiar el comportamiento de esta molécula en membranas modelo de \textit{Staphylococcus aureus}, compuestas por DMPG o DPPG utilizando simulaciones por dinámica molecular.
\section{Objetivos Específicos}
\begin{enumerate}
\item Reproducir las simulaciones por dinámica molecular de estafiloxantina en DMPG y DPPG realizadas  por Meléndez et al. \cite{MelendezDelgado2018StudyingBilayers}, utilizando los parámetros no optimizados de la cadena diaponeurosporenoica.
\item Optimizar los parámetros relacionados  con los ángulos dihedros de la cadena diaponeurosporenoica próximos al enlace éster. Posteriormente, realizar simulaciones por dinámica molecular de estafiloxantina embebida en membranas de DMPG y DPPG.
\item Generar un campo de fuerzas para estafiloxantina utilizando los parámetros de AMBER y realizar simulaciones por dinámica molecular con este campo de fuerzas. Esto con el fin de demostrar que el comportamiento de la molécula no depende del tipo de potencial utilizado.
%\item De las simulaciones de stafiloxantina en membranas DMPG y DPPG, obtener los parámetros biofísicos: área por lípido, ángulo de orientación de la estafiloxantina, parámetro de orden del deuterio, coeficiente de difusión y espesor de la membrana.
\item Examinar el efecto de la concentración de estafiloxantina sobre las propiedades de la bicapa lipídica, realizando simulaciones de dinámica molecular con parámetros generados en los numerales 2 y 3, aumentando  la concentración de estafiloxantina al 15\%.
\end{enumerate}

